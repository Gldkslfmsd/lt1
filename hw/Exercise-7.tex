%%%%%%%%%%%%%%%%%%%%%%%%%%%%%%%%%%%%%%%%%
% Short Sectioned Assignment
% LaTeX Template
% Version 1.0 (5/5/12)
%
% This template has been downloaded from:
% http://www.LaTeXTemplates.com
%
% Original author:
% Frits Wenneker (http://www.howtotex.com)
%
% License:
% CC BY-NC-SA 3.0 (http://creativecommons.org/licenses/by-nc-sa/3.0/)
%
%%%%%%%%%%%%%%%%%%%%%%%%%%%%%%%%%%%%%%%%%

%----------------------------------------------------------------------------------------
%	PACKAGES AND OTHER DOCUMENT CONFIGURATIONS
%----------------------------------------------------------------------------------------

\documentclass[paper=a4, fontsize=11pt]{scrartcl} % A4 paper and 11pt font size
\usepackage[margin=1in]{geometry}
\usepackage[T1]{fontenc} % Use 8-bit encoding that has 256 glyphs
%\usepackage{fourier} % Use the Adobe Utopia font for the document - comment this line to return to the LaTeX default
\usepackage[english]{babel} % English language/hyphenation
\usepackage{amsmath,amsfonts,amsthm} % Math packages
\usepackage{enumerate}
\usepackage{hyperref}
\usepackage{sectsty} % Allows customizing section commands
\allsectionsfont{\centering \normalfont\scshape} % Make all sections centered, the default font and small caps
\urldef\paper\url{http://arxiv.org/pdf/1301.3781.pdf}
\urldef\wordvec\url{https://code.google.com/p/word2vec/}
\usepackage{fancyhdr} % Custom headers and footers
\usepackage{listings}
\pagestyle{fancyplain} % Makes all pages in the document conform to the custom headers and footers
\fancyhead{} % No page header - if you want one, create it in the same way as the footers below
\fancyfoot[L]{} % Empty left footer
\fancyfoot[C]{} % Empty center footer
%\fancyfoot[R]{\thepage} % Page numbering for right footer
\renewcommand{\headrulewidth}{0pt} % Remove header underlines
\renewcommand{\footrulewidth}{0pt} % Remove footer underlines
\setlength{\headheight}{13.6pt} % Customize the height of the header

\numberwithin{equation}{section} % Number equations within sections (i.e. 1.1, 1.2, 2.1, 2.2 instead of 1, 2, 3, 4)
\numberwithin{figure}{section} % Number figures within sections (i.e. 1.1, 1.2, 2.1, 2.2 instead of 1, 2, 3, 4)
\numberwithin{table}{section} % Number tables within sections (i.e. 1.1, 1.2, 2.1, 2.2 instead of 1, 2, 3, 4)

\setlength\parindent{0pt} % Removes all indentation from paragraphs - comment this line for an assignment with lots of text

\usepackage{xcolor,colortbl}
\definecolor{Gray}{gray}{0.85}

%----------------------------------------------------------------------------------------
%	TITLE SECTION
%----------------------------------------------------------------------------------------

\newcommand{\horrule}[1]{\rule{\linewidth}{#1}} % Create horizontal rule command with 1 argument of height

\title{	
\vspace{-1.1cm}
\normalfont \normalsize 
\textsc{Language Technology I, Winter 2016-2017} \\ [25pt] % Your university, school and/or department name(s)
%\horrule{0.5pt} \\[0.4cm] % Thin top horizontal rule
\huge Exercise 7: Word Vectors \\ % The assignment title
%\horrule{2pt} \\[0.5cm] % Thick bottom horizontal rule
}

%\author{John Smith} % Your name

\date{} % Today's date or a custom date

\begin{document}

\maketitle % Print the title

%----------------------------------------------------------------------------------------
%	PROBLEM 1
%----------------------------------------------------------------------------------------

\vspace{-2cm}
\textit{You can earn up to 10 points on this exercise.\\
You may work as a group of up to 3 people, but please submit your own version.\\
You may use any programming language you wish.}\\

\textit{Please email your solution as a single PDF file to} \texttt{langtech1saarlandws1617@gmail.com} \textit{by} \textbf{3:00 PM GMT$+1$, January 18, 2017}.\\


%\vspace{3cm}

We will soon send you a private link to a large text data set via the
\texttt{langtech1saarlandws1617@gmail.com} email address using the
email addresses we've collected.  Download the data and do the
following:

\begin{enumerate}
\item The data consists of thousands of documents.  Randomly select
  and extract enough documents to generate at least 2 million
  words. This is our corpus.  Lowercase and tokenize them (you can use
  the script from Exercise 4 or any other method).  Report the unigram
  frequencies of the top 50 words in a table. (2 points)
\item Arbitrarily choose 15-20 words for fruit in the corpus. Give the
  unigram frequency table. Then arbitrarily choose 15-20 words for
  junk food (single word, avoid brand names if possible) in the
  corpus.  Give the unigram frequency table. (1 point)
\item Construct word vectors by any algorithm you like using any
  programming environment you prefer, so long as they're at least
  100-dimensional.  Present, aligned, 100 dimensions of one of your
  fruit words and one of your junk food words. (1 point)
\item Project the vector space down to two dimensions using any
  projection you like (SVD, PCA, t-SNE, etc.). Then present the following,
  stating which projection you use:
  \begin{enumerate}[(a)]
  \item A plot of the fruit food vectors. (2 points)
  \item A plot of the junk food vectors. (2 points)
  \end{enumerate}
  Make sure the labels are easy to read.
\item Use K-means clustering (there are packages in Python and other
  languages that will do this for you) to cluster the fruit and junk
  food vectors {\it together}.  Force it to make 5 clusters, with
  whichever other parameters you like.  For each cluster, list the
  members. (2 points)
\end{enumerate}



\end{document}
