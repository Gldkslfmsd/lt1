%%%%%%%%%%%%%%%%%%%%%%%%%%%%%%%%%%%%%%%%%
% Short Sectioned Assignment
% LaTeX Template
% Version 1.0 (5/5/12)
%
% This template has been downloaded from:
% http://www.LaTeXTemplates.com
%
% Original author:
% Frits Wenneker (http://www.howtotex.com)
%
% License:
% CC BY-NC-SA 3.0 (http://creativecommons.org/licenses/by-nc-sa/3.0/)
%
%%%%%%%%%%%%%%%%%%%%%%%%%%%%%%%%%%%%%%%%%

%----------------------------------------------------------------------------------------
%	PACKAGES AND OTHER DOCUMENT CONFIGURATIONS
%----------------------------------------------------------------------------------------

\documentclass[paper=a4, fontsize=11pt]{scrartcl} % A4 paper and 11pt font size
\usepackage[margin=1in]{geometry}
\usepackage[T1]{fontenc} % Use 8-bit encoding that has 256 glyphs
%\usepackage{fourier} % Use the Adobe Utopia font for the document - comment this line to return to the LaTeX default
\usepackage[english]{babel} % English language/hyphenation
\usepackage{amsmath,amsfonts,amsthm} % Math packages
\usepackage{enumerate}
\usepackage{hyperref}
\urldef\myhomepage\url{http://www.coli.uni-saarland.de/courses/CL/2015/exercises/Exercise-8.dat}
\usepackage{sectsty} % Allows customizing section commands
\allsectionsfont{\centering \normalfont\scshape} % Make all sections centered, the default font and small caps

\usepackage{fancyhdr} % Custom headers and footers
\pagestyle{fancyplain} % Makes all pages in the document conform to the custom headers and footers
\fancyhead{} % No page header - if you want one, create it in the same way as the footers below
\fancyfoot[L]{} % Empty left footer
\fancyfoot[C]{} % Empty center footer
%\fancyfoot[R]{\thepage} % Page numbering for right footer
\renewcommand{\headrulewidth}{0pt} % Remove header underlines
\renewcommand{\footrulewidth}{0pt} % Remove footer underlines
\setlength{\headheight}{13.6pt} % Customize the height of the header

\numberwithin{equation}{section} % Number equations within sections (i.e. 1.1, 1.2, 2.1, 2.2 instead of 1, 2, 3, 4)
\numberwithin{figure}{section} % Number figures within sections (i.e. 1.1, 1.2, 2.1, 2.2 instead of 1, 2, 3, 4)
\numberwithin{table}{section} % Number tables within sections (i.e. 1.1, 1.2, 2.1, 2.2 instead of 1, 2, 3, 4)

\setlength\parindent{0pt} % Removes all indentation from paragraphs - comment this line for an assignment with lots of text

\usepackage{xcolor,colortbl}
\definecolor{Gray}{gray}{0.85}

%----------------------------------------------------------------------------------------
%	TITLE SECTION
%----------------------------------------------------------------------------------------

\newcommand{\horrule}[1]{\rule{\linewidth}{#1}} % Create horizontal rule command with 1 argument of height

\title{	
\vspace{-1.1cm}
\normalfont \normalsize 
\textsc{Language Technology I, Winter 2016-2017} \\ [25pt] % Your university, school and/or department name(s)
%\horrule{0.5pt} \\[0.4cm] % Thin top horizontal rule
\huge Exercise 1: Formal languages and grammars \\ % The assignment title
%\horrule{2pt} \\[0.5cm] % Thick bottom horizontal rule
}

%\author{John Smith} % Your name

\date{} % Today's date or a custom date

\begin{document}

\maketitle % Print the title

%----------------------------------------------------------------------------------------
%	PROBLEM 1
%----------------------------------------------------------------------------------------

\vspace{-2cm}
\textit{You can earn up to 10 points on this exercise.\\
%You may work as a group of up to 3 people, but please submit your own version.\\
%You may use any programming language you wish, but any submission that we cannot run on our computers without installing things must be presented to the class.
}\\

\textit{Please email your solution to} \texttt{langtech1saarlandws1617@gmail.com} \textit{with the subject header} \textbf{Exercise 1} \textit{by} \textbf{15:00, November 9, 2016}.\\

In the following grammars, strings within double quotes (``\ldots'') denote terminal symbols.  \\All others are non-terminals.

\vspace{3cm}

\section*{Task 1}

Consider the following grammar:

\begin{table}[h!]
\begin{center}
\begin{tabular}{c c c c}
S  & $\rightarrow$ & N & ``fish''\\
N  & $\rightarrow$ & \multicolumn{2}{c}{``fish''}\\
\end{tabular}
\end{center}
\end{table}

\begin{enumerate}[a.]
\item What kind(s) of formal language does the grammar admit? Explain your answer. (1 point)
\item Add one rule such that the above grammar represents a non-finite, non-regular, context-free language. Explain the relevant properties of your rule. (1 point)
%\item Describe the general form of the sentences belonging to the part (b) language. (1 point)
\end{enumerate}

\newpage

\section*{Task 2}

Now consider the following grammar:

\begin{table}[h!]
\begin{center}
\begin{tabular}{c c c c}
S  & $\rightarrow$ & NP  & VP\\
NP & $\rightarrow$ & Det & N \\
PP & $\rightarrow$ & P   & NP\\
VP & $\rightarrow$ & V   & NP\\
   &               &     &   \\
Det & $\rightarrow$ & \multicolumn{2}{c}{``the''}\\
N   & $\rightarrow$ & \multicolumn{2}{c}{``fish''}\\
P   & $\rightarrow$ & \multicolumn{2}{c}{``with''}\\
V   & $\rightarrow$ & \multicolumn{2}{c}{``moved''}\\
\end{tabular}
\end{center}
\end{table}

\begin{enumerate}[a.]
\item What kind(s) of formal language does the grammar admit? Explain your answer. (1 point)
\item Add one \textit{sensible} rule to the grammar such that the grammar admits a non-finite, non-deterministic context-free language. Explain the relevant properties of your rules. (1 point)
\item Describe the general form of the sentences belonging to the part (b) language. (1 point)
%\item Add one \textit{sensible} rule to the grammar such that the grammar admits a non-finite, \textit{non-}deterministic context-free language. Explain the relevant properties of your rules. (1 point)
%\item What does it mean that the grammar is in Chomsky Normal Form (CNF)? What kinds of languages can be described using grammars in CNF? Add a few (perhaps non-sensical) rules to the grammar that are context-free, but not in CNF. (1 point)
\end{enumerate}

\vspace{3cm}

\section*{Task 3}

\begin{enumerate}[a.]
\item What is the general form of context-sensitive grammar rules and why are they necessary to describe cross serial dependencies in Swiss German? (1 point)
\item Consider questions in English. Provide one reason why transformations might be useful in describing the grammar of English questions,
and one disadvantage of the transformation approach. (2 points) 
%\item Briefly explain the transformation system for affix hopping, as described in Noam Chomsky's \textit{Syntactic Structures} (1957).  What level of formal language is required to implement this system? (1 point)
\end{enumerate}

\vspace{3cm}

\section*{Task 4}

\begin{enumerate}[a.]
\item Write down a meaning for the English word ``chair'' in terms of between 5 and 10 binary features. (1 point)
\item Express the sentence ``the teacher sat on the chair'' in first-order predicate calculus. (1 point)
\end{enumerate}

\end{document}
